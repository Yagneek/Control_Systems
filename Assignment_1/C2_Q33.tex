\documentclass{beamer}
\mode<presentation>
\usepackage{amsmath}
\usepackage{amssymb}
%\usepackage{advdate}
\usepackage{adjustbox}
\usepackage{subcaption}
\usepackage{enumitem}
\usepackage{multicol}
\usepackage{listings}
\usepackage{url}
\def\UrlBreaks{\do\/\do-}
\usetheme{Dresden}
\usecolortheme{beaver}
\setbeamertemplate{footline}

{
  \leavevmode%
  \hbox{%
  \begin{beamercolorbox}[wd=\paperwidth,ht=2.25ex,dp=1ex,right]{author in head/foot}%
    \insertframenumber{} / \inserttotalframenumber\hspace*{2ex} 
  \end{beamercolorbox}}%
  \vskip0pt%
}
\setbeamertemplate{navigation symbols}{}

\providecommand{\nCr}[2]{\,^{#1}C_{#2}} % nCr
\providecommand{\nPr}[2]{\,^{#1}P_{#2}} % nPr
\providecommand{\mbf}{\mathbf}
\providecommand{\pr}[1]{\ensuremath{\Pr\left(#1\right)}}
\providecommand{\qfunc}[1]{\ensuremath{Q\left(#1\right)}}
\providecommand{\sbrak}[1]{\ensuremath{{}\left[#1\right]}}
\providecommand{\lsbrak}[1]{\ensuremath{{}\left[#1\right.}}
\providecommand{\rsbrak}[1]{\ensuremath{{}\left.#1\right]}}
\providecommand{\brak}[1]{\ensuremath{\left(#1\right)}}
\providecommand{\lbrak}[1]{\ensuremath{\left(#1\right.}}
\providecommand{\rbrak}[1]{\ensuremath{\left.#1\right)}}
\providecommand{\cbrak}[1]{\ensuremath{\left\{#1\right\}}}
\providecommand{\lcbrak}[1]{\ensuremath{\left\{#1\right.}}
\providecommand{\rcbrak}[1]{\ensuremath{\left.#1\right\}}}
\theoremstyle{remark}
\newtheorem{rem}{Remark}
\newcommand{\sgn}{\mathop{\mathrm{sgn}}}
\providecommand{\abs}[1]{\left\vert#1\right\vert}
\providecommand{\res}[1]{\Res\displaylimits_{#1}} 
\providecommand{\norm}[1]{\lVert#1\rVert}
\providecommand{\mtx}[1]{\mathbf{#1}}
\providecommand{\mean}[1]{E\left[ #1 \right]}
\providecommand{\fourier}{\overset{\mathcal{F}}{ \rightleftharpoons}}
%\providecommand{\hilbert}{\overset{\mathcal{H}}{ \rightleftharpoons}}
\providecommand{\system}{\overset{\mathcal{H}}{ \longleftrightarrow}}
	%\newcommand{\solution}[2]{\textbf{Solution:}{#1}}
%\newcommand{\solution}{\noindent \textbf{Solution: }}
\providecommand{\dec}[2]{\ensuremath{\overset{#1}{\underset{#2}{\gtrless}}}}
\newcommand{\myvec}[1]{\ensuremath{\begin{pmatrix}#1\end{pmatrix}}}
\let\vec\mathbf

\lstset{
%language=C,
frame=single, 
breaklines=true,
columns=fullflexible
}

\numberwithin{equation}{section}

\title{Assignment 1}
\author{Yagneek\\EE19BTECH11031}

\date{\today} 
\begin{document}

\begin{frame}
\titlepage
\end{frame}


\begin{frame}
\tableofcontents
\end{frame}
\section{Problem}
\begin{frame}
\frametitle{Chapter 2, Problem 33}
For the rotational system shown in the below figure, find the transfer function,
\begin{equation*}
    G(s) = \frac{\theta_2(s)}{T(s)}.
\end{equation*}

\begin{figure}
\centering
\includegraphics[width=0.8\columnwidth]{figs/question_pic.png}
\end{figure}

\end{frame}

%\subsection{Literature}
\section{Solution}
\begin{frame}
\frametitle{Solution}
%\framesubtitle{Literature}
Since the inertias are tied together through gears, they do not undergo linearly independent motion. Thus, there
is only one degree of freedom and hence one equation of motion.\\ \ \\
    \quad We can solve this problem using the concept of reflection of impedances. We first reflect the impedances and torque on the first shaft to the second shaft. Then, we reflect the impedances on the fourth shaft to the third shaft.
    The resulting system will not have any gears and we can find its transfer function by using the fact that :

\begin{align*}
    (Sum \ of \ impedances)(Angular \ displacement) = Sum \ of \ applied \ torques
\end{align*}
\end{frame}
\begin{frame}

We know that, rotational mechanical impedances can be reflected through gear trains by multiplying the
mechanical impedance by the ratio :

\begin{align*}
    \left( \frac{Number \ of \ teeth \ of \ gear \ on \ destination \ shaft}{Number \ of \ teeth \ of \ gear \ on \ source \ shaft} \right)^2
\end{align*}

where the impedance to be reflected is attached to the source shaft and is being reflected to the destination shaft.

\end{frame}
\begin{frame}
Similarly, we also know that torques can be reflected through gear trains by multiplying the torque by the ratio :

\begin{align*}
    \left( \frac{Number \ of \ teeth \ of \ gear \ on \ destination \ shaft}{Number \ of \ teeth \ of \ gear \ on \ source \ shaft} \right)
\end{align*}

where the source and destination shafts are defined in a similar way.

\end{frame}
\begin{frame}

After completing the reflections, we see that the equivalent impedances and torque are as follows :

\begin{align*}
    
    D_E = \left( \frac{N_2}{N_1} \right)^2 D_1 + D_2 + \left( \frac{N_3}{N_4} \right)^2 D_3\\
    
    K_E = \left( \frac{N_3}{N_4} \right)^2 K\\
    
    J_E = \left( \frac{N_2}{N_1} \right)^2 J_1 + J_2 + \left( \frac{N_3}{N_4} \right)^2 J_3\\
    
    T_E = \left( \frac{N_2}{N_1} \right)T(s)
    
\end{align*}
\end{frame}

\begin{frame}
    
    Substituing the values of
    \begin{align*}
        N_1 = 4, \ N_2 = 12, \ N_3 = 4, \ N_4 = 16\\
        J_1 = 2, \ J_2 = 1, \ J_3 = 16\\
        D_1 = 1, \ D_2 = 2, \ D_3 = 32\\
        K = 64
    \end{align*}
    we find that,
    \begin{align*}
        D_E = 13\\
        K_E = 4\\
        J_E = 20\\
        T_E = 3T(s)
    \end{align*}
    
\end{frame}

\begin{frame}
    
    Since the equivalent system has no gears, has impedances of J\textsubscript{E},D\textsubscript{E} and K\textsubscript{E} and a torque of T\textsubscript{E} acting on it, the laplace transform of the equation of motion will be :
    
    \begin{align*}
        (J_Es^2+D_Es+K_E)\theta_2(s) = T_E
    \end{align*}
    Substituting the calculated values, the transfer function will be
    
    \begin{align*}
        G(s) = \frac{\theta_2(s)}{T(s)} = \frac{3}{20s^2+13s+4}
    \end{align*}
    
\end{frame}
\section{Code}

\begin{frame}[fragile]
\frametitle{Github link}
    The python code is available here
    {\footnotesize
\begin{lstlisting}
https://github.com/Yagneek/Control_Systems/blob/master/Assignment_1/C2_P33.py
\end{lstlisting}
}
\end{frame}

\section{Plot}
\begin{frame}[fragile]
\frametitle{Impulse Response}
\begin{figure}
\centering
\includegraphics[width=0.9\columnwidth]{figs/plot_1.png}
\label{fig:plane}
\end{figure}
\end{frame}

\begin{frame}[fragile]
\frametitle{Step Response}
\begin{figure}
\centering
\includegraphics[width=0.9\columnwidth]{figs/plot_2.png}
\label{fig:plane}
\end{figure}
\end{frame}
%\begin{frame}
%\frametitle{Introduction}
%\framesubtitle{Literature}
%%\begin{figure}[t!]
%%    \centering
%%    \begin{subfigure}[t]{0.4\columnwidth}
%%        \centering
%%        \includegraphics[width=\columnwidth]{point_source}
%%        \caption{Single point source}
%%\label{fig3:subfig1}        
%%    \end{subfigure}%
%%    ~ 
%%    \begin{subfigure}[t]{0.4\columnwidth}
%%        \centering
%%        \includegraphics[width=\columnwidth]{pointNoPowerDist_new}
%%        \caption{SNR profile}
%%\label{fig3:subfig2}
%%    \end{subfigure}
%%  %  \caption{Average SNR for a BPP. $N=16$}
%%    \label{fig3}
%%  \end{figure}
%
%\end{frame}
%  
%
%
%%

\end{document}

